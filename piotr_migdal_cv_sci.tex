
%________________________________________________________________________________________
% @brief    LaTeX2e Resume for Piotr Migdał
% @author   Piotr Migdał
% @url      http://migdal.wikidot.com/en
% @date     November 2009
% @info     Based on Latex Resume Template by Chris Paciorek 
%           http://www.biostat.harvard.edu/~paciorek/

%________________________________________________________________________________________
\documentclass[margin,line]{resume}
\topmargin -5mm

\usepackage[utf8]{inputenc}
\usepackage[plmath,OT4]{polski}

\RequirePackage{color,graphicx}
\usepackage[usenames,dvipsnames]{xcolor}
\usepackage{hyperref}
\definecolor{linkcolour}{rgb}{0,0.2,0.6}
%\definecolor{linkcolour}{rgb}{0,0,0}
\hypersetup{colorlinks,breaklinks,urlcolor=linkcolour, linkcolor=linkcolour}
%\usepackage{fontspec} 
%\defaultfontfeatures{Scale=MatchLowercase,Mapping=tex-text} % pozwala na -- i ,,''
%\setromanfont{Gentium}

\newcommand{\superscript}[1]{\ensuremath{^{\textrm{#1}}}}
\newcommand{\thh}[0]{\superscript{th}}
\newcommand{\st}[0]{\superscript{st}}
\newcommand{\nd}[0]{\superscript{nd}}
\newcommand{\rd}[0]{\superscript{rd}}

%\sectionwidth{10}
% resumewidth	

\begin{document}
%\title{Curriculum Vitae}
%\name{\Large Curriculum Vitae}
\name{\Large Curriculum Vitae}

\begin{resume}

    %____________________________________________________________________________________
    % Personal Information
    \section{\mysidestyle Personal\\Information}\vspace{2mm}

    \begin{tabular}{@{} l @{\hspace{28mm}} l}
    First name / Surname:    & Piotr Migdał             \\
    Date of birth:           & 1986-03-13               \\
    Citizenship:             & Polish                   \\
    E-mail:                  & \href{pmigdal@gmail.com}{\tt pmigdal@gmail.com}        \\
    Homepage:			& \href{http://migdal.wikidot.com}{\tt http://migdal.wikidot.com} \\
    Phone:                   & +34 644 226 536 (Spanish) / +48 537 459 068 (Polish)         \\
    % Phone:                  & +1 (415) 231-8022 (US), +34 644 226 536 (Spain) \\
    Profiles: & \href{https://github.com/stared}{GitHub}, \href{http://stackexchange.com/users/506817/piotr-migdal?tab=accounts}{StackExchange}\\
 %   Address:                 & ul. Łęgowa 28            \\
%                             & 43-300 Bielsko-Biała     \\
%                             & Poland
    \end{tabular}

%\vspace{3mm}

    %____________________________________________________________________________________
    % Objective
%    \section{\mysidestyle Objective}
%    A competitive Ph.D. programme in theoretical physics (or related discipline), compatible with the research interests.
    
\vspace{3mm}

    %____________________________________________________________________________________
    % Research Interests
    \section{\mysidestyle Research\\Interests}
    complex systems, complex networks, data science, mathematical modelling in psychology,
    geometry of quantum states, quantum optics, quantum information

\vspace{3mm}

    %____________________________________________________________________________________
    % Education
    \section{\mysidestyle Education}
    
    {\bf ICFO - The Institute of Photonic Sciences},  Castelldefels (Barcelona), Spain \\%
    Quantum Optics Theory Group, advisor: prof. Maciej Lewenstein \hfill {\bf Feb 2011 -- }\\
   \begin{list2}
        \vspace*{-4mm}
        \item Symmetries and self-similarities of quantum states, application of sequence-analysis methods for quantum states.
        \item Complex quantum networks: walks, community detection.
        \item Collaboration: mathematical psychology.
    \end{list2}

    {\bf University of Warsaw}, Warsaw, Poland \\
    {\sl Inter-faculty Studies in Mathematics and Natural Sciences} \hfill {\bf 2005 -- 2011}\\
    \begin{list2}
      \vspace*{-4mm}
      \item Master's Degree  (5 year programme) {\hfill Jan 2011}\\
      Physics (Theoretical Physics --- Quantum Optics and Atomic Physics)\\
      thesis: \href{http://migdal.wikidot.com/en:collective-decoherence}{Quantum codes immune to collective decoherence and photon loss},\\
      advisor: Konrad Banaszek, grade: 5/5
      %(thesis: {\sl A generalized transfer-matrix method in optics and quantum mechanics}) 
      \item Bachelor's Degree, Mathematics \hfill Sept 2009\\
      thesis: \href{http://migdal.wikidot.com/en:mafia}{A mathematical model of the mafia game}, advisor: Jacek Miekisz, grade: 5/5
    \end{list2}
    
    % {\bf High School No 5}, Bielsko-Biała, Poland \\%
    % {\sl Programme with extended mathematics, physics and computer science} \hfill {\bf 2001 -- 2005}\\
    %  graduated with first-class honours degree

%\vspace{3mm}
\newpage

    %____________________________________________________________________________________
    % Research Experience
    \section{\mysidestyle Research\\Experience}

    {\bf  Startup Compass Inc.}, San Francisco, CA, USA\\
    Data Science Intern \hfill {\bf Jul -- Oct 2013}\\
    \begin{list2}
        \vspace*{-4mm}
        \item Designing distance function for software companies, based on a multidimensional dataset.
        \item Data processing (Python: SciPy, Pandas; MongoDB), data exploration and visualization (Gephi, D3.js).
    \end{list2}

    {\bf  ISI Foundation}, Turin, Italy\\
    Jacob Biamonte Gorup \hfill {\bf May -- Jun 2013}\\
    \begin{list2}
        \vspace*{-4mm}
        \item Quantum complex networks: quantum walks, community detection, weak time-symmetry.
    \end{list2}

	{\bf  Interdisciplinary Centre for Mathematical and Computational Modelling, University of Warsaw}, Warsaw, Poland\\
	Dariusz Plewczyński Group, mathematical psychology \hfill {\bf Nov 2010 -- }\\
    \begin{list2}
        \vspace*{-4mm}
        \item Cognitive computing: human information sharing models, models of human perception.
    \end{list2}

	{\bf Institute of Theoretical Physics, University of Warsaw}, Warsaw, Poland\\
	Konrad Banaszek Group, quantum optics (theoretical physics) ,
	TEAM Programme operated by the Foundation for Polish Science \hfill {\bf Jan -- Sep 2010}\\
    \begin{list2}
        \vspace*{-4mm}
        \item Quantum-enhanced protocols in realistic environments: Generation schemes for robust entangled states.
    \end{list2}

    {\bf ICFO - The Institute of Photonic Science},\\ Castelldefels (Barcelona), Spain\\
    Maciej Lewenstein Group, quantum optics (theoretical physics) \hfill {\bf Oct 2009}\\
    \begin{list2}
        \vspace*{-4mm}
        \item Ultra-cold atomic gases in non-abelian gauge fields.
    \end{list2}

    {\bf Faculty of Physics, Nicolaus Copernicus University}, Toruń, Poland\\
    Konrad Banaszek Group, quantum optics (theoretical physics) \hfill {\bf 2007 -- 2009}\\
    \begin{list2}
        \vspace*{-4mm}
        \item Analysis of spontaneous parametric down-conversion, estimation of the quantum noise.
        \item Averaging procedures for generating decoherence-free states.
    \end{list2}

    {\bf Faculty of Physics, University of Warsaw}, Warsaw, Poland\\
    Czesław Radzewicz Group, applied optics (experimental physics) \hfill {\bf 2006 -- 2007}\\
    \begin{list2}
        \vspace*{-4mm}
        \item Wavefront sensor with Fresnel zone plates for use in an undergraduate laboratory.
    \end{list2}

    {\bf Francis Bitter Magnet Laboratory, Massachusetts Institute of Technology},\\
    Jagadeesh Moodera Group, spintronics (experimental physics) \hfill Cambridge (MA), USA\\
    during the Research Science Institute 2005 scholarship \hfill {\bf Jul –- Aug 2005}\\
    \begin{list2}
        \vspace*{-4mm}
        \item Experimental work on organic tunnelling barriers.
    \end{list2}

\vspace{3mm}
\newpage

    %____________________________________________________________________________________
    % Reference
%    \section{\mysidestyle References}

%The two most recent collaborators:\vspace{3mm}\\
% \begin{tabular}{@{} l @{\hspace{4mm}} l @{\hspace{8mm}} l@{\hspace{4mm}} l }
%prof. & Czesław Radzewicz       & czeslaw.radzewicz@fuw.edu.pl  & (academic advisor, collaborator)   \\
%dr    & Wojciech Wasilewski     & wojciech.wasilewski@fuw.edu.pl & (collaborator)  \\

%    \end{tabular}
    
%\vspace{3mm}


    %____________________________________________________________________________________
    % Activities
    \section{\mysidestyle Projects,\\Activities}
    Teaching, popularization and organizational work
        \begin{list2}
        \item Started an unconference series \href{http://offtopicarium.wikidot.com/en:start}{Offtopicarium} \hfill { Jan 2012--}
	    \item a moderator of \href{http://theoreticalphysics.stackexchange.com/}{Theoretical Physics - Stack Exchange}\hfill Nov 2011--May 2012
        \item prepared and lead a month-length course {\sl Introduction to quantum cryptography} for 2 talented high-school students for Caixa Catalunya (Jovenes y Ciencia) \hfill Jun 2011, Jul 2012
        \item co-organizer of the \href{http://warsztatywww.wikidot.com/en}{6th and 7th  Summer Scientific Schools} \hfill 2010-2011%\\
        \item co-organizer of \href{http://www.flaszki.waw.pl/}{Flaszki} (a series of 5-min talks) \hfill 2010
%        (non-profit camp for gifted high school students)
        \item member of the {\sl Neurobiology Students' Scientific Society}, Univ. of Warsaw  \hfill 2008--2011
        \item voluntary tutor of the \href{http://www.fundusz.org/?lang=gb}{Polish Children's Fund} during 9 scientific workshops \hfill 2006--2011\\
        (for gifted high school individuals), the last one: \href{http://migdal.wikidot.com/fizyka-stada}{Physics of Herd}
        \item 9 talks on students' conferences (physics, mathematics, psychology) \hfill 2006--2010
        \item 5 scientific and didactic shows \hfill 2006--2010
        \item mentor of A. Kubica and W. Pilewski, winners (1st prize) of the \href{http://www.eucys09.fr/}{21st European Union Contest for Young Scientists in Paris}  \hfill 2009
%            in the {\sl Polish Finals of the European Union Contest for Young Scientists} for its participants (Aleksander Kubica, Wiktor Pilewski) and in 1st in the 21st {\sl EUCYS} in Paris
        \item head organizer of the 7th Polish Physics Students' Societies Conference \hfill 2007--2009\\
        (7--10.11.2008, University of Warsaw, 80 participants from 15 universities, 36 talks)
        \item co-founder and president of the \href{http://skfiz.fuw.edu.pl/en}{Physics Students' Society}, Univ. of Warsaw \hfill 2006--2009
         \item problem setter of the Polish Physics Olympiad \hfill 2006--2008
        \end{list2} 
	
	Courses and conferences --- participant 
	        \begin{list2}
        \item \href{http://bigdive.eu}{BigDive} (a 4-week long hands-on workshop in data science and big data processing, for 20 participants), Turin, Italy \hfill {2012}
        \item \href{http://www.uibk.ac.at/th-physik/qism2012/}{Quantum Information meets Statistical Mechanics}, Innsbruck, Austria \hfill 2012
        \item \href{http://www.eccs2012.eu/}{ECCS12}: European Conference on Complex Systems, Brussels, Belgium \hfill 2012
        \item \href{http://qcmc2012.org/}{QCMC2012}: 11th Intl. Conference on Quantum Communication, Measurement and Computing, Vienna, Austria \hfill 2012
        \item \href{http://www.fizyka.umk.pl/zfmis/smp44/}{44 Symposium on Mathematical Physics: New Developments in the Theory of Open Quantum Systems}, Toruń, Poland  \hfill 2012
	    \item \href{http://www.eccs2011.eu/}{ECCS2011}: European Conference of Complex Systems, including events: \href{http://markov.uc3m.es/complexdynamics11/Home.html}{Complex Dynamics of Human Interactions} and \href{http://cssociety.org/PhDVienna2011}{PhD 'Research in Progress' Workshop}, Vienna, Austria  \hfill 2011
	    \item \href{http://netsci2011.net/}{NetSci 2011}: The International School and Conference on Network Science\hfill 2011
	    \item \href{http://bss.mafihe.hu/}{Balaton Summer School in Physics}: Self-organization and complex systems, Hungary \hfill 2010
        \item \href{http://www.icoam.org/}{International Conference on Optical Angular Momentum}, York, UK \hfill 2010
        \item {Summer Course: Quantum Engineering, Advanced Level}, Warsaw, Poland \hfill 2009
 %       (31.08-25.09.2009, Faculty of Physics, University of Warsaw)
        \item \href{http://www.cft.edu.pl/QuantumOpticsVII/}{Quantum Optics VII - Quantum Engineering of Atoms and Photons} Zakopane, Poland \hfill 2009
        \item {Quantum Optics and Quantum Information}, Toruń, Poland \hfill 2008 
 %       (22-25.08.2008, Faculty of Physics, Nicolaus Copernicus University, Toruń)  
 %       \item 7 scientific workshops organized by the \href{http://www.fundusz.org/?lang=gb}{Polish Children's Fund} \hfill 2003--2005
                \end{list2} 
       
     Other
	        \begin{list2}
	    \item co-founder of \href{http://confrenzy.com}{Confrenzy} --- a website listing scientific events \hfill 2011--2012
        % \item Data Science intern at Startup Compass Inc. --- \hfill Jul--Oct 2013
                \end{list2} 

%jeszcze wWw!

    %courses
    %students conferences
    %conferences
    %didactics?

\vspace{3mm}

   %____________________________________________________________________________________
    % Awards and Scholarships
    \section{\mysidestyle Awards,\\Scholarships}
    \begin{list2}
    	\item ICFO Innovation Fund grant awarded for Confrenzy \hfill Oct 2011
        \item Scholarship of the Minister of Science and Higher Education\\for exceptional achievements in science \hfill 2007--2010
        \item 2nd  place in the Didactic Show Competition (Faculty of Physics, University of Warsaw)  \hfill 2008  %6.12.2008 
        \item Research Science Institute (Massachusetts Institute of Technology, Cambridge, MA, USA) \hfill 2005
        \item Scholarship of the Polish Children's Fund \hfill 2003--2005
        \item Scholarship of the Minister of Education, Science and Sport \hfill 2003--2005
        \item Bronze Medal in the International Physics Olympiad (Pohang, South Korea) \hfill 2004
        \item 2nd  place in the 53rd Polish Physics Olympiad \hfill 2004
        \item 7th  place in the 52nd Polish Physics Olympiad \hfill 2003
    \end{list2}

\vspace{3mm}

    %____________________________________________________________________________________
    % Skills
    \section{\mysidestyle Technical skills}
    \begin{list2}
        \item Languages: Polish (native), English (fluent)
        \item Programming languages: Mathematica, Python, MATLAB, LabView, LaTeX, JavaScript
        \item Systems (user): Mac OS X, Linux, Windows
    \end{list2}

%\vspace{3mm}
\newpage

    %____________________________________________________________________________________
    % Awards and Scholarships
    \section{\mysidestyle Publications}
    Papers
    \begin{list2}
        \item P. Migdał, K. Rodríguez-Laguna, M. Oszmaniec, M. Lewenstein,
        \href{http://arxiv.org/abs/1403.3069}{Multiphoton states related via linear optics?}, Phys. Rev. A 89, 062329 (2014), arXiv:1403.3069,
        featured in the Editor's Suggestions of Physical Review A. 
        \item M. Faccin, P. Migdał, T. Johnson, J. Biamonte, V. Bergholm,
        \href{http://arxiv.org/abs/1310.6638}{Community Detection in Quantum Complex Networks}, arXiv:1310.6638
        \item M. Faccin, T. Johnson, J. Biamonte, S. Kais, P. Migdał,
        \href{http://arxiv.org/abs/1305.6078}{Degree Distribution in Quantum Walks on Complex Networks},
        Phys. Rev. X 3, 041007 (2013),
        arXiv:1305.6078,
        featured on \href{http://johncarlosbaez.wordpress.com/2013/08/05/quantum-network-theory-part-1/}{Azimuth blog}
        \item P. Migdał, J. Rodriguez-Laguna, M. Lewenstein,
        \href{http://arxiv.org/abs/1305.1506}{Entanglement classes of permutation-symmetric qudit states: symmetric operations suffice},
        Phys. Rev. A 88, 012335 (2013), arXiv:1305.1506
        \item J. Rodriguez-Laguna, P. Migdał, M. Ibanez Berganza, M. Lewenstein, G. Sierra, \href{http://dx.doi.org/10.1088/1367-2630/14/5/053028}{Qubism: self-similar visualization of many-body wavefunctions}, New J. Phys. 14 053028 (2012), arXiv:1112.3560,
        in \href{http://iopscience.iop.org/1367-2630/page/Highlights%20of%202012}{the NJP Highlights of 2012}
		\item P. Migdał, M. Denkiewicz, J. Rączaszek-Leonardi, D. Plewczynski, \href{http://arxiv.org/abs/1109.2044}{Information-sharing and aggregation models for interacting minds},
        Journal of Mathematical Psychology 56, 417-426 (2012),
        arXiv:1109.2044,
        \href{http://egtheory.wordpress.com/2014/01/30/two-heads-are-better-than-one-how-about-more/}{blog post}
        \item P. Migdał, K. Banaszek, \href{http://arxiv.org/abs/1107.3786}{Immunity of information encoded in decoherence-free subspaces to particle loss}, Phys. Rev. A 84, 052318 (2011), arXiv:1107.3786
    	\item P. Migdał, \href{http://arxiv.org/abs/1009.1031}{A mathematical model of the Mafia game}, arXiv:1009.1031
        \item P. Migdał, W. Wasilewski, \href{http://dx.doi.org/10.1007/s00340-010-3915-z}{Noise reduction in 3D noncollinear parametric amplifier}, Appl. Phys. B 99, 657-671 (2010), arXiv:0908.2207
        \item K. Banaszek, R. Demkowicz-Dobrzański, M. Karpinski, P. Migdał, C. Radzewicz, \href{http://arxiv.org/abs/0908.3548}{Quantum and semiclassical polarization correlations}, Opt. Comm. 283, 713-718 (2010), arXiv:0908.3548
        \item P. Migdał, P. Fita, C. Radzewicz, Ł. Mazurek, \href{http://ultrafast.fuw.edu.pl/publications/ajp_2008.pdf}{Wavefront sensor with Fresnel zone plates for use in an undergraduate laboratory}, Am. J. Phys. 76, 229 (2008)
        \item T. S. Santos, J. S. Lee, P. Migdal, I. C. Lekshmi, B. Satpati, J. S. Moodera, \href{http://dx.doi.org/10.1103/PhysRevLett.98.016601}{Room-Temperature Tunnel Magnetoresistance and Spin-Polarized Tunneling through an Organic Semiconductor Barrier}, Phys. Rev. Lett. 98, 016601 (2007)
    \end{list2}

    Conference talks --- presenting author
     \begin{list2}
        \item P. Migdał, M. Denkiewicz, J. Rączaszek-Leonardi, D. Plewczynski, {\sl Two and more heads deciding: models of information-sharing and aggregation for two-choice discriminative tasks}, \href{http://markov.uc3m.es/complexdynamics11/Home.html}{Complex Dynamics of Human Interactions} (a \href{http://www.eccs2011.eu/}{ECCS2011} satellite) (12-16.09.2011, Vienna, Austria)
    \end{list2}   

    Conference posters --- presenting author
    \begin{list2}
        \item J. Rodriguez-Laguna, P. Migdał, M. Ibanez Berganza, M. Lewenstein, G. Sierra, \href{http://dx.doi.org/10.6084/m9.figshare.97233}{Self-similar visualization and sequence analysis of many-body wavefunctions} (2012)
        \item \href{http://dx.doi.org/10.6084/m9.figshare.97235}{Immunity of information encoded in singlet states against one particle loss} (2012)
        \item A. Kubica, P. Migdał, \href{http://dx.doi.org/10.6084/m9.figshare.97236}{The spatial shape of the Spiral Zone Plate foci} (2010)
        \item P. Migdał, W. Wasilewski, \href{http://dx.doi.org/10.6084/m9.figshare.97237}{Optimization of a 3D noncollinear parametric amplifier} (2009)
    \end{list2}
    
    Popular science and education-related articles (selected)
    \begin{list2}
        \item M. Kotowski, M. Kotowski, P. Marczewski, P. Migdał, \href{http://warsztatywww.wikidot.com/en:indie-camp-for-hs-geeks}{An independent camp for high school geeks}, Summer Scientific School (2012)
        \item M. Kotowski, P. Migdał, \href{http://offtopicarium.wikidot.com/v1:open-science-2-0}{Open Science and Science 2.0}, 1st Offtopicarium (2012)
        \item P. Migdał, S. Krawczyk, \href{http://migdal.wikidot.com/local--files/nerdowska-duma/Migdal_Krawczyk__Zespol_Aspergera_preprint_after_vkkk.pdf}{Zespół Aspergera, nauki ścisłe i kultura nerdów} (Asperger Syndrome, Hard Science and Nerd Culture), V Krakowska Konferencja Kognitywistyczna (2011), a popular science article
		% \item P. Migdał, \href{http://www.epigen.arabidopsis.pl/swn09/calosc.pdf}{Szczypta magii w każdym promyku – o polaryzacji światła} (A bit of magic in every ray - on the polarization of light), III Sylwestrowe Warsztaty Naukowe, 65-67 (2010), a popular science article
		\item M. Kotowski, M. Kotowski, P. Migdał, K. Sołtys, \href{http://warsztatywww.wikidot.com/drogowskaz-pasjonata}{Drogowskaz Pasjonata, czyli jak rozwijać się w szkole i w trakcie studiów} (Guidelines for the Curious - how to develop oneself during high school and university years), (2010), a collection of advice
		\item P. Migdał, \href{http://www.mimuw.edu.pl/delta/artykuly/delta2010-07/2010-07-6.pdf}{Mafia, zdradziecka parzystość oraz pi} (Mafia, treacherous parity and pi), Delta - miesięcznik popularnonaukowy, 14-15, 07/2010 (2010), a short popular science article
		\item P. Migdał, \href{http://migdal.wikidot.com/zapalency-i-wypalency}{Zapaleńcy i Wypaleńcy, czyli rzecz o utracie pasji w trakcie studiów} (Flames of passion… and of burnout, or: about the loss of motivation during studies), (2010), an essay
		\item eds: M. Zientkiewicz, P. Migdał, M. Nowaczyk, M. Pomorski, B. Szczygieł, \href{http://skfiz.fuw.edu.pl/vii-osknf:proceedings}{7th Polish Physics Students' Societies Conference - proceedings}, ISBN: 978-83-61026-05-1, (2009), a book (132 pages, 10 reviewed papers)
    \end{list2}



% \vspace{3mm}
%     %____________________________________________________________________________________
%     % Hobbies
%     \section{\mysidestyle Hobbies}
%     photography, hiking, cognitive science

%________________________________________________________________________________________
\end{resume}
\end{document}

%________________________________________________________________________________________
% EOF
